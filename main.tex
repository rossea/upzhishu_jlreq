\documentclass[uplatex,dvipdfmx,tate,book,twoside,openright,onecolumn,hanging_punctuation,
% •paper=[< 紙サイズ名 >/{< 寸法 >,< 寸法 >}] :紙サイズです.紙サイズ名は a0 から a10 , b0 か
%  ら b10 , c2 から c8 を指定できます. B 列は JIS B 列です.また, {< 横 >,< 縦 >} と直接寸法を
%  指定することもできます.
paper=b6,
% •fontsize=< 寸法 ;Q,H> :欧文フォントサイズ.デフォルトは 10pt .
% •jafontsize=< 寸法 ;Q,H> :和文フォントサイズ.
% •jafontscale=< 実数値 > :欧文フォントと和文フォントの比(和文 / 欧文). fontsize と
%  jafontsize が両方指定されている場合は無視される.デフォルトは 1 .
% jafontscale=0.924690,
% •line_length=< 寸法 ;zw,zh> :一行の長さ.デフォルトは字送り方向の紙幅の 0.75 倍.実際
%  の値は一文字の長さの整数倍になるように補正されます.
line_length=40zh,
% •number_of_lines=< 自然数値 > :一ページの行数.デフォルトは行送り方向の紙幅の 0.75 倍
%  になるような値.
number_of_lines=15,
% •gutter=< 寸法 ;zw,zh> :のどの余白の大きさ.
%  – tate 無指定時は奇数ページ左,偶数ページ右の余白
%  – tate 指定時は奇数ページ右,偶数ページ左の余白
%  – twoside が指定されていない時は,常に奇数ページ扱いで余白が設定される
% gutter=4zw,
% •fore-edge=< 寸法 ;zw,zh> :小口(のどでない方)の余白の大きさ.「日本語組版処理の要件」
%  にある方法で余白を指定する限り使われることはありませんが,便利なこともあるので実装さ
%  れています.
% •head_space=< 寸法 ;zw,zh> :天の空き量.デフォルトは中央寄せになるような値.
% head_space=15zw
% •foot_space=< 寸法 ;zw,zh> :地の空き量.デフォルトは中央寄せになるような値.
% •baselineskip=< 寸法 ;Q,H,zw,zh> :行送り.デフォルトは jafontsize の 1.7 倍.
% •linegap=< 寸法 ;Q,H,zw,zh> :行間.
linegap=0.75zw, 
% •headfoot_sidemargin=< 寸法 ;zw,zh> :柱やノンブルの左右の空き.
headfoot_sidemargin=12zw,
% •column_gap=< 寸法 ;zw,zh> :段間( twocolumn 指定時のみ).
column_gap=0zh,
% •sidenote_length=< 寸法 ;zw,zh> :傍注の幅を指定します.
% sidenote_length=8zw, % 頭注的高度
open_bracket_pos=nibu_tentsuki,%open_bracket_pos=zenkakunibu_nibu
]{jlreq}

\usepackage{pjlreq} %自定义jlreq.cls设置
\usepackage{fontsettings} %自定义虚拟字体的设置


\begin{document}
% maketitle を使わずに独自のタイトルページを作る
\begin{titlepage}
  \vspace*{10mm}
  \noindent{\fontsize{30pt}{48pt}\mcfamily 书籍标题}
  \vfill

  \begin{flushright}
    {\mcfamily\huge 著者\quad 朝代\quad 作者}
  \end{flushright}
\end{titlepage}

\setlength{\parindent}{2.4zw}
%\phantomsection
%%%%%%%%%%%%测试代码开始
\AddEverypageHook{%此后每页都添加网格
  \BgMaterial
}
%%%%%%%%%%%%测试代码结束
\addcontentsline{toc}{chapter}{前言}
\chapter*{前言}
\fontsize{12pt}{17.1}\selectfont
\zhlipsum[1-5][name=trad]
\fontsize{12pt}{15}\selectfont
%ここには序の内容が入る。
\tableofcontents % 目次
\clearpage
\pagestyle{plain}
\setlength{\parindent}{2.4zw}
\Large
\chapter{字符空间测试}
\input{chinese.tex}

\pagestyle{plain}
\chapter{章节目录一}
\zhlipsum[1-5][name=trad]
\chapter{最初の章}
ここは最初の章の冒頭の文章が入る。
ここは最初の章の冒頭の文章が入る。
\section{最初の節の見出し}
ここは最初の節の文章が入る。
ここは最初の節の文章が入る。
ここは最初の節の文章が入る。
ここは最初の節の文章が入る。
ここは最初の節の文章が入る。
\section{第二の節の見出し}
ここは第二の節の文章が入る。
ここは第二の節の文章が入る。
ここは第二の節の文章が入る。
ここは第二の節の文章が入る。
ここは第二の節の文章が入る。

\chapter{便利な命令}
\section{文字装飾}
  \bou{傍点}・\kenten{圏点}・\kasen{傍線}
\section{特殊文字など}

%   \ajMaru{0} \ajMaru{5} \ajMaru{42} \ajMaru{100}
%   \ajMaru*{0} \ajMaru*{5} \ajMaru*{42} \ajMaru*{100}
%   \ajKuroMaru{0} \ajKuroMaru{5} \ajKuroMaru{42} \ajKuroMaru{100}
%   \ajKuroMaru*{0} \ajKuroMaru*{5} \ajKuroMaru*{42} \ajKuroMaru*{100}

%   \ajKaku{0} \ajKaku{5} \ajKaku{42} \ajKaku{100}
%   \ajKaku*{0} \ajKaku*{5} \ajKaku*{42} \ajKaku*{100}
%   \ajKuroKaku{0} \ajKuroKaku{5} \ajKuroKaku{42} \ajKuroKaku{100}
%   \ajKuroKaku*{0} \ajKuroKaku*{5} \ajKuroKaku*{42} \ajKuroKaku*{100}

%   \ajMaruKaku{0} \ajMaruKaku{5} \ajMaruKaku{42} \ajMaruKaku{100}
%   \ajMaruKaku*{0} \ajMaruKaku*{5} \ajMaruKaku*{42} \ajMaruKaku*{100}
%   \ajKuroMaruKaku{0} \ajKuroMaruKaku{5} \ajKuroMaruKaku{42} \ajKuroMaruKaku{100}
%   \ajKuroMaruKaku*{0} \ajKuroMaruKaku*{5} \ajKuroMaruKaku*{42} \ajKuroMaruKaku*{100}


% 「こら〳〵」「どれ〴〵」

% 参加者は\,\rensuji{12}\,人だった。


\section{注釈}

脚注\footnote{脚注。}を表示する。

後注\endnote{これが後注の文章である。}を表示する。

割注\warichu{これが割注の文章である。}を表示する。


% \section{漢文}

% % \kundoku{漢字}{ルビ}{送り仮名}{返り点}[肩返り点]<左送り仮名>(句読点)

% \kundoku{未}{いま}{ダ}{レ}<ル>
% \kundoku{知}{}{ラ}{二}
% 仁
% \kundoku{義}{}{ヲ}{一}
% \kundoku{也}{}{}{}(。)

\end{document}